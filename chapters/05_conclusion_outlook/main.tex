\chapter{Conclusion and Future Work}
\label{ch:Conclusion}

\graphicspath{{chapters/05_conclusion_outlook/figures/}}
% \dictum[Vincent van Gogh]{%
%    I dream my painting and I paint my dream. }
% \vskip 1em

\section{Summary}
This thesis has expanded the current use of urban modeling and simulation to inform urban policy-making in cities. It explored innovative approaches to address existing challenges in urban areas through advanced modeling and simulation techniques in three case studies across various spatial scales. Each case study presented in Chapters~\ref{ch:VR}-\ref{ch:ETRCO2H} has contributed with novel methodologies and unique insights that collectively advance the understanding and practice of urban planning and policy-making in the built environment, embracing social-centered and dynamic processes.

In Chapter~\ref{ch:VR}, at a small spatial scale focused on streetscape, a subject-based VR experiment investigated the pedestrian perception of future street scenarios. The findings from the 2x3 factorial experiment underscored the importance of inclusive and democratic urban design processes, revealing preferences for familiar solutions and highlighting the impact of communication strategies on public endorsement of innovative street layouts. Communication and behavioral aspects have a greater impact than purely technocratic and tangible aspects of design and efficiency-oriented advantages.

The medium urban spatial scale centered on the public space network and urban fabric is distributed in two chapters. Chapter~\ref{ch:getting_real} focused on the technical challenge of building and validating a large-scale digital twin of traffic for the Metropolitan Region of Barcelona using real data. This agent-based microsimulation provided a robust platform for exploring citywide effects of planning and policy scenarios, demonstrating the potential of such models in enhancing urban policy decision-making and participatory processes. Chapter~\ref{ch:less_can_be_more} precisely uses that microsimulation platform to evaluate the impact of reducing street space for private vehicles in Barcelona, following the current plans of the city within the world-known Superblocks paradigm. Using agent-based simulation, the study identified positive counterintuitive effects resulting in a global reduction of congestion and emission despite the reduction of space for private vehicles, which is aligned with the hard-to-anticipate, evasive Braess and Daganzo paradoxes.  These outcomes challenge conventional urban planning and policy assumptions, also common arguments used in public participatory commissions with citizens, to advocate for sustainable, participatory city-making practices able to disentangle complex urban phenomena.

At a larger urban scale connected to land use and mobility in the metropolitan and territorial domain, Chapter~\ref{ch:ETRCO2H} introduces a spatial econometric approach to estimate the impact on traffic OD flows in the RMB caused by the adaptive conversion of offices into housing. Particularly, it proposed a novel approach based on spatial lags of the independent variables applied to OD pairs and a solving approach based on applying gradient boosting on a two-part Hurdle model. The pluralistic modeling approach points out potential global traffic reductions despite the induced population densification, showcasing the utility of these models to enrich current processes of internal city renovation in parallel with new production and organizational trends in the workplace such as teleworking and remote working.

\section{Societal impacts}

The new methodologies and findings presented in this thesis demonstrate the feasibility and benefits of using advanced modeling and simulation techniques in urban planning and policy-making in cities. Overall, the research promotes evidence-based decision-making and enhances public engagement and the consideration of diverse agendas and opinions in urban policy formulation and design. The emphasis on shedding light on the complexity found in the built environment, particularly resulting from the interactions with humans aims to foster a more democratic and inclusive approach to city-making, aligned with contemporary trends towards sustainable, resilient, and people-centered urban development.



Through visiting different topics and urban scales, this thesis embarked on a comprehensive revisit of urban modeling and simulation, how they are relevant to current challenges in cities, and particularly the value they can have in participatory decision-making processes. This operability and interactivity expands these approaches and  relates them directly to urban digital twins across various spatial scales, methodologies, disciplines, and applications to support resilient, sustainable, and participatory practices in city-making:

% NEW
The importance of location-based approaches is fundamental in the development of theory and models aiming at describe and understand human settlement and activities distribution. Intervening in urban shape and structure is assumed to improve quality of life in cities through urban planning, design, and policy. However, this approach risk dismissing qualitative aspects, sometimes aspatial although strongly located and situated, linked to socio-economic, cultural, human systems, which are even more determining of urban life. % humans not only as studied compoentns, but active constituent components of the twins

\begin{itemize}
    \item Expanding urban digital twins: the thesis advances the concept of urban digital twins as sophisticated, interactive, near real-time, digital representations of cities. Through VR human-based experiments, large-scale traffic microsimulations, and spatial econometric modeling the thesis prototypes operational components that can be integrated as part of urban digital twins to tackle complex urban phenomena and respond to actual challenges. Urban digital twins can potentially be also crucial for the evaluation of SDGs in voluntary national and local reviews (VNRs and VLRs).
    \item Questioning and reformulating disciplinary boundaries: this thesis reframes the relationship between different stakeholders and professionals involved in city-making processes. It highlights the collaborative and interdisciplinary work performed by urban planners, architects, designers, policymakers, technologists, social scientists, community leaders, businesses, developers, and land owners. Hence, it advocated for a transition from prescriptive top-down urban planning and policy-making to a collaborative, iterative, multi-dependent, and potentially adaptive one. The proposed systems can support interdisciplinary approaches that blend technical expertise with social insights to co-create cities that meet diverse societal needs and aspirations.
    \item Technical and operational advancements: the thesis elaborates on existing techniques and expands with novel approaches to model, analyze, manage, and anticipate urban dynamics, otherwise hard to grasp. By providing critical, counterintuitive insights into the potential impacts of urban interventions at different scales before their implementation, these techniques can improve the decision-making process. 
    \item Participation and stakeholder engagement: the thesis emphasizes the importance of transparent, explainable, understandable, and accountable methods used for modeling as fundamental conditions for participatory urban digital twins. In each case study, the applications tried to provide support methods to consider alternative visions of the city and not as a mere optimization problem to solve. Rather the opposite, they invite to explore alternative futures of cities and their impacts through open-ended frameworks whose value is not in their predictive power \emph{per se} but in their exploratory possibilities. Also, they aim to be accessible. They do not require necessarily expert knowledge to use them although they allow for sophisticated modeling. Hence, any stakeholder should be able to express, share, test, and confront their ideas and opinions, which could be a valuable resource as well for citizen science and digitally-assisted deliberation and democracy.
    \item Challenging established assumptions: the thesis expresses a consistent and constant effort to challenge, question, and even falsify common assumptions and misconceptions about what could be the impact of certain decisions or interventions in the built environment. By presenting counterintuitive insights, unexpected outcomes, or just simply contrary to general beliefs, this research encourages a re-evaluation of traditional urban planning paradigms. Furthermore, it can support the development of new theories regarding urban systems from a socio-technical perspective and in connection to human behavior in the built environment. In general, it highlights the potential for innovative solutions that might otherwise be overlooked and the development of policy and planning supported by data, evidence --even though it could be only via simulated prognosis-- and taking into account complexity science.
\end{itemize}

\section{Future Work}

Looking ahead, the thesis concludes by highlighting its methodological value for untangling complex effects of the built environment, with a practical impact in planning and policy-making, to enhance and foster participation, diversity, and crowd-knowledge building from a societal point of view. Hence, future research can follow these lines:

On one hand, it is possible to point out specific improvements for each case study, relative to their particular methodology and scope. Additionally to the limitations, improvement and future work addressed in each chapter, it is possible to highlight the following:
\begin{itemize}
    \item Expanding the VR experiments to cover other built environments and different social groups to improve our knowledge about how different socio-demographic groups perceive and experience different spatial and cultural settings. % in the shoes of others
    \item Agent-based microsimulation can incorporate other types of transportation and real-time data, to enhance functionality and predictive capabilities. The former requires further improvement, particularly regarding the modeling of pedestrian behavior and navigation in urban environments, which is an open research question. The latter would allow us to interact with more dynamic, changing situations. 
    \item Continuing the refinement of spatial econometric models by incorporating more data, and new approaches regarding the specification of spatial interactions.
    
\end{itemize}

A comprehensive digital twin of physical cities should encompass all spatio-temporal scales, integrate data flows, be owned and serve a purpose for people, and be integrated into the ethical and governance frameworks. This is the real challenge of such systems from a socio-technical point of view, ranging from technical to societal and cultural challenges.

\begin{itemize}
    \item Scalability and Interoperability: One of the frequently identified challenges of urban digital twins is scalability and interoperability with an increasing amount of data sources, platforms, and models. The development and use of standards that can be used in many different contexts and situations is crucial. Also, a modular approach based on the development of building blocks and components. This can be combined with the development of new, more flexible, and richer encoding approaches, based on new semantics that could simultaneously capture any type of interaction and element from reality, and enhance faster computation.
    \item Enhanced socio-technical integration: Challenging the technocratic visions of smart cities and urban modeling, there is an increasing interest in incorporating social, economic, cognitive, and perceptual aspects of cities that need to be reinforced and prioritized. Among others, it requires the inclusion of qualitative data, community feedback mechanisms, and participatory design principles into these frameworks.
    \item Ethical and governance frameworks: If modeling and urban digital twins are expected to be integrated with urban governance, and design processes, and particularly convey stakeholders' opinions through participation, the development of robust ethical guidelines, and legal and governance frameworks becomes imperative. These frameworks should safeguard privacy, ensure data security and integrity, promote fairness in decision-making, and stay transparent and accountable, thereby building trust and legitimacy among stakeholders in cities.
    
\end{itemize}
% scalability
% generalization
Even more generally, a lot of effort needs to be put into creating and testing theories and models across different locations and contexts. 

\section{Data-driven limits}

The promise of urban digital twins inherited from the early concepts developed for manufacturing and aerospace development, also applicable to models and simulations in general, is that more data will provide necessarily higher accuracy, and then better predictions. However, these computational approaches aiming to mirror physical reality face challenges connected to limits of predictability, limits of coupling, and more fundamentally limits of data representation and computing. Data is bounded, partial, mediated, and curated by definition, which therefore limits the way we represent the complex urban environments, which are not deterministic machines.

It does not mean that the massive amount of high-quality data is not able to build good models. We have also progressed a lot in the development of near real-time exchange information processes with a very small time lag. Also, we have gained valuable experience in articulating fair and effective participatory processes. Nevertheless, there are still plenty of challenges to be overcome. From the governance point of view, political authority and controversies for having the legitimacy to define what is the common good and prioritize goals are factors that need to be taken into account. From a technical and data perspective, it is needed to take into account ownership, agency, and accessibility to data. From a participatory perspective, the risk for flawed forms of participation and public engagement, and the challenges of leveraging power and knowledge difference across stakeholders need to be incorporated in the design of these systems.

Alternatively, the thesis wants to illustrate examples at different scales of implementations of urban modeling that bridge the gap between analysis and planning or design challenges relevant to current cities. As such, these implementations could be components of larger urban digital twins that should assist in city functioning and planning at many spatio-temporal scales and in an endless number of domains. Thereby, these urban digital twins should mirror cities, and hence, provide support in open-ended planning processes, as cities are by themselves. Beyond deterministic predicting machines, these technologies can enable and facilitate the sharing of diverse opinions and agendas, while virtual sandboxes would be valuable for testing new policies safely. In the end, urban digital twins need to be conceived and understood within the very nature of cities, and hence they are more valuable as exploration tools for what our built environment could be rather than as mere super accurate predictors of very well-defined, and also narrow, futures.

\section*{Epilogue: Disclaimer for a uncertain possibilistic future\textsuperscript{*}}
\addcontentsline{toc}{section}{Epilogue: Disclaimer for a uncertain possibilistic future}

\footnotetext{* This epilogue has been published as part of the book chapter: \fullcite{ArgotaSanchez-Vaquerizo2023_3Tales}.}

The future is hard to predict. Most likely, it will be different from any of these tales. Some of the elements will resonate in future cities, but in different ways and with different interactions and implementations. Each of these new technologies will have effects on culture, society, and ultimately the environment that are hard to anticipate \citep{Berkhout2004, Gao2014}. Some of them will be undesirable too. They are not complete visions either, as they do not lay out every detail. Nevertheless, all of these visions agree on the importance of the exploration of `what-if’ scenarios as planning tools to envision more resilient cities supported by conversations and discovery processes with computer intelligence. These approaches match architecture and urban planning aim to change the way we think about how things work and how things could be in alternative scenarios rather than how they are now  \citep{Doucet2009, Simon1969}. There are alternative futures to be explored in this conversational relationship with virtual versions of our environment \citep{Pask1976} beyond purely physical-digital interactive feedback for management \citep{Fuller2020}, endless detailed 1:1 mirrors \citep{Borges1946}, solely data-driven approaches \citep{Caldarelli2023, VanDijck2014}, and transhumanist escapes from the physical realm \citep{Kye2021}. 

%% Too much focus on form. More focus needed on aspatial and non-spatial, although strongly situated, socio-economic and cultural nehavioral processes.