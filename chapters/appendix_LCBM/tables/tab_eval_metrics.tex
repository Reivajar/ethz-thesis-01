% \usepackage{tabularray}
\begin{longtblr}[
  caption = {Description of metrics used to analyse the results of the simulations.},
  label = {tab:appendix_LCBM_eval_metrics}
]{
  width = \linewidth,
  cells = {font=\fontsize{8}{9.6}\selectfont},
  colspec = {Q[225] Q[827]},
  cell{2}{1} = {c=2}{0.942\linewidth},
  cell{6}{1} = {c=2}{0.942\linewidth},
  cell{13}{1} = {c=2}{0.942\linewidth},
}
\textbf{Metric (unit)}                      & \textbf{Explanation and optimization criteria}                                                                                                                                                                                                                                                                                                                                                                                                                   \\
\textbf{Simulation performance}             &                                                                                                                                                                                                                                                                                                                                                                                                                                                                  \\
Vehicles loaded factor                    & Ratio between vehicles programmed into the transportation demand of the baseline scenario and the transportation demand of the alternative scenario. A \textbf{value closer to 1} represents a transportation demand closer to the original baseline demand.  \\
\# vehicles inserted                      & Total number of vehicles inserted in the simulation. Vehicles programmed in the original transportation demand, which are not inserted in the simulation, are due to high levels of traffic saturation at the expected entry time. A \textbf{higher }number of vehicles inserted represents a better distribution of traffic and a lower level of congestion, which avoids skipping vehicles to be inserted.                                                     \\
Total teleports (as \% of total vehicles) & Teleporting is a mechanism that SUMO uses to avoid agents (e.g. vehicles or pedestrians) to indefinitely get stuck in the simulation, moving them to the following free road section of their route if they were stopped for longer than a specified time. A \textbf{lower} proportion of teleporting is a sign of a more realistic simulation and lower congestion, as it does not need to rely on this somewhat artificial mechanism to keep the agents going. \\
\textbf{Efficiency}                         &                                                                                                                                                                                                                                                                                                                                                                                                                                                                  \\
Average departure delay (s)               & Average time that a vehicle had to wait before starting their journeys. This is caused due to the lack of space in the starting location for inserting the vehicle caused by traffic congestion. A \textbf{lower} value represents a more realistic traffic scenario and is assumed to be preferred by the users.                                                                                                                                                \\
Average trip duration (s)                 & Average trip duration of the vehicles. A \textbf{lower} travel time is a sign of better mobility performance due to lower congestion, shorter routes, and/or higher speeds, and is assumed to be preferred by the users.                                                                                                                                                                                                                                         \\
Average trip length (m)                   & Average route length of the vehicles. A \textbf{shorter} average route length implies availability of more direct routes, which are not congested, leading to shorter travel times, and is assumed to be preferred by the users.                                                                                                                                                                                                                                 \\
Average speed (m/s)                       & Average trip speed of the vehicles. A \textbf{higher} average speed (i.e. closer to the speed limit of the network) may be caused by lower levels of congestion and means fewer stops, shorter travel times, and less fuel consumption due to a more steady speed, and is assumed to be preferred by the users.                                                                                                                                                  \\
Average time loss (s)                     & Average time loss due to driving slower than the desired speed (it includes waiting time as well). A \textbf{lower} time loss means less traffic congestion, higher average speed, and shorter travel times, and is assumed to be preferred by the users.                                                                                                                                                                                                        \\
Average waiting time (s)                  & Average time spent standing involuntarily (i.e. speed below 0.1 m/s). A \textbf{lower} waiting time is caused by less congestion and means faster trips, less wasted time, and less fuel. Hence, it is assumed to be preferred by the users.                                                                                                                                                                                                                     \\
\textbf{Environmental}                      &                                                                                                                                                                                                                                                                                                                                                                                                                                                                  \\
Total CO\textsubscript{2} (tons/day)                      & Total emissions of carbon dioxide (CO2) by all the vehicles in the simulation. \textbf{Lower} values are preferred.                                                                                                                                                                                                                                                                                                                                              \\
Total CO (tons/day)                       & Total emissions of carbon monoxide (CO) by all the vehicles in the simulation. \textbf{Lower} values are preferred.                                                                                                                                                                                                                                                                                                                                              \\
Total HC (tons/day)                       & Total emissions of unburnt hydrocarbons (HC) by all the vehicles in the simulation. \textbf{Lower} values are preferred.                                                                                                                                                                                                                                                                                                                                         \\
Total NO\textsubscript{x} (tons/day)                      & Total emissions of nitrogen oxides (NOX) by all the vehicles in the simulation. \textbf{Lower} values are preferred.                                                                                                                                                                                                                                                                                                                                             \\
Total PM\textsubscript{x} (tons/day)                      & Total emissions of particulate matter (PMX) by all the vehicles in the simulation. \textbf{Lower} values are preferred.                                                                                                                                                                                                                                                                                                                                          \\
Total fuel (millions of litres/day)       & Total fuel consumption by all the vehicles in the simulation. \textbf{Lower} values are preferred.                                                                                                                                                                                                                                                                                                                                                               
\end{longtblr}